% NÃO altere as predefinições desse template!

\documentclass{ceel}

% ===========================
%Coloque aqui pacotes adicionais, se necessário
\usepackage{verbatim}
\usepackage{subfig}
%\usepackage{caption}
%\usepackage{subcaption}
\usepackage{hyperref}%
\usepackage{array}
%\usepackage{wrapfig}
%%===========================

% Dados do trabalho
\title{Métodos de limiarização de imagem QR Code em distintos padrões de luminosidade}

% Autores: o primeiro será, necessariamente, o apresentador do trabalho
% Caso o trabalho não tenha 8 autores, exclua os campos que não foram preenchidos
\author[1]{\underline{Lesly Viviane Montúfar Berrios}\thanks{leslymontufar@ufu.br}}
\author[2]{Luciano Xavier Medeiros\thanks{lucianox@ufu.br}}
\author[2]{Alexandre Coutinho Mateus\thanks{acmateus@ufu.br}}


% Adicione as instituições de cada autor e indique corretamente no campo acima
\affil[1]{FEELT - Universidade Federal de Uberlândia}
\affil[2]{FEELT - Professor Adjunto - Universidade Federal de Uberlândia}

\begin{document}

\inserirtitulo

\begin{multicols}{2}

% Adicione o Resumo do seu trabalho no campo abaixo, com início em "O objetivo [...]"
\textbf{\emph{Resumo} - Este artigo propõe três métodos de binarização para imagens de QR Codes sujeitos a diferentes padrões de luminosidade e, subsequentemente, a análise de eficiência de cada técnica desenvolvida. Sabe-se que, intrínseco ao processo de reconhecimento do um símbolo QR, é necessária a aplicação de uma técnica de binarização, a qual é influenciada pela iluminação ambiente. Sob essa perspectiva, é de interesse avaliar quais seriam os métodos ótimos considerando-se as peculiaridades do ambiente de captura da imagem.}
\vspace*{10pt}

%Adicione as palabras-chave do seu trabalho abaixo
\textbf{\emph{Palavras-Chave} - Análise de eficiência, binarização, imagem, luminosidade, QR Code.}


\begin{center}
%Insira aqui o Título do trabalho em inglês
\noindent\textbf{\large \uppercase{Thresholding methods of QR Code images in different brightness patterns}}
\end{center}

%Insira aqui o resumo do seu trabalho em inglês 
\textbf{\emph{Abstract} - This paper proposes three binarization methods for QR Codes images subjected to different light patterns and, subsequently, the efficiency analysis of each developed technique. It is known that, intrinsic to the recognition process of a QR symbol, it is necessary to apply a binarization technique, which is influenced by ambient lighting. From this perspective, it is of interest to evaluate which methods would be optimal considering the peculiarities of the image capture environment.} 
\vspace*{10pt}

%Insira aqui as palavras-chave do seu trabalho em inglês
\textbf{\emph{Keywords} - Brightness, efficiency analysis, image, QR Code, thresholding.}

%Introdução: Caso queira pode mudar o título da seção para qualquer outro. Dentro das chaves insira o título da seção e abaixo insira o texto da mesma.
\section{Introdução}
Com surgimento no Japão em 1994, os Quick Response Codes (QR Codes) são marca registrada da empresa Denso-Wave Incorporated, uma subsidiária da Toyota, cujo propósito de criação estava em facilitar o processo de catalogação de componentes automobilísticos \cite{denso}. Entretanto, sua utilização não se ateve a essa finalidade, devido à grande velocidade em que podem ser lidos e decodificados eletronicamente. Além disso, é uma ferramenta que tem sido bastante difundida mundialmente, devido ao rápido desenvolvimento de dispositivos portáteis capazes de ler QR Codes e ao crescente uso comercial, principalmente associado à área de marketing.

Em termos práticos, a utilização de telefones celulares, câmeras digitais ou outros equipamentos para a coleta da imagem de código QR pode apresentar empecilhos significantes para o reconhecimento do símbolo, devido tanto a fatores do ambiente, quanto ao manuseio do usuário. Sob essa perspectiva, o ambiente de captura mostra-se como aspecto de bastante relevância e está associado, prinicipalmente, a irregularidades na iluminação. Outrossim, o manuseio do equipamento pelo usuário pode ocasionar a distorção geométrica, problemática que pode ser minimizada, como propõe, por exemplo, Jin et al. \cite{j-chen}, por meio de um algoritmo de correção de imagem de código QR baseado na transformada wavelet, a qual consiste basicamente na decomposição dos dados de forma a possibitar a análise tanto do domínio da frequência quanto do tempo.

A temática deste artigo, no entanto, contempla os obstáculos devido a padrões de iluminação irregulares. Ademais, atentar-se, deve, acerca do tipo de imagem que se espera trabalhar, ou seja, comumente de baixo contraste, o que é relevante na escolha do processamento adequado. Por isso, idealizou-se três métodos baseados na limiariação, os quais foram aplicados e recolhidos dados sobre imagens QR Code obtidas em padrões de iluminação diferentes. Dessa maneira, propõe-se verificar o nível de correspondência de imagens de código QR submetidas a métodos de binarização distintos, para assim poder determinar o mais eficiente e estabelecer uma relação com o ambiente de captura.

Na Seção \ref{caracteristicas} serão apresentadas características básicas dos QR Codes. Na Seção \ref{metodologia}, o procedimento realizado para a coleta de imagens e exemplificação do impacto provocado pela iluminação irregular Já na Seção \ref{metodos}, são descritos e exemplificados os algoritmos das técnicas de binarização desenvolvidas, assim como realizada a análise de eficiência para as imagens do exemplo. Finalmente, na Seção \ref{resultados}, discute-se acerca dos resultados obtidos com os métodos propostos neste artigo.


\section{Características dos Códigos QR} \label{caracteristicas}
O codificação QR baseia-se na disposição de pequenos quadrados, normalmente de cores preta e branca. Esses são denominados módulos e possuem certa dimensão em píxel. Além disso, o tamanho do código é determinado pelo número de versão, considerando que a Versão 1 possui dimensão de $21\times21$ módulos e para as versões subsequentes há o acréscimo linear de 4 módulos, ou seja, a Versão 2 possui dimensões de 25 x 25, a Versão 3 tem 29 x 29, e assim por diante. 

Apresenta também uma técnica de correção de erro, a qual permite a decodificação e, por conseguinte, obtenção da informação contida na imagem, mesmo que se encontre danificada, devido a inserção de dados redundantes ao código. Logo quanto maior a capacidade de correção, menor o espaço para o armazenamento de dados. A Tabela \ref{nivel-correcao} descreve os 4 possíveis níveis de correção de erro.

\vspace{0.15cm}
\begin{minipage}[h]{\columnwidth}
\begin{scriptsize}
    \def\arraystretch{1.35}
    \captionsetup{type=table}
    \begin{center}
    \caption{Níveis de Correção de Erro.} \label{nivel-correcao}
    \vspace{-0.2cm}
    \begin{tabular}{ c | c } \hline
    \textbf{Nível de correção de erro}&  \textbf{Capacidade de correção}\\\hline
     L & Recupera 7\% dos dados \\
    M & Recupera 15\% dos dados \\
    Q & Recupera 25\% dos dados \\
    H & Recupera 30\% dos dados \\\hline
    \end{tabular}
    \end{center}
\end{scriptsize}
\end{minipage}
\vspace{0.3cm}

Conforme a Figura \ref{regioes}, a estrutura do código constitui-se de duas principais regiões: \textit{região de codificação} e \textit{elementos funcionais}. A região de codificação inclui a \textit{Format Information} e \textit{Version Information}, além das \textit{Data Codewords} e \textit{Error Correction Codewords}. Já os elementos funcionais são compostos pelo \textit{Finder Patterns}, \textit{Separator}, \textit{Timing Patterns} e \textit{Alignment Patterns}. Assim, quando uma imagem de QR Code é capturada, busca-se por esses padrões para o reconhecimento e decodificação rápida dos dados \cite{qr-tutorial}.

\vspace{0.25cm}
\hspace{-0.45cm}
\begin{minipage}[h]{\columnwidth}
%\captionsetup{type=figure, singlelinecheck=off, justification=raggedright, margin=0pt, font=footnotesize}
\captionsetup{type=figure, margin=5pt}
\includegraphics[width=\columnwidth]{regioes-qr-upercase}
\caption{\label{regioes}Regiões de um QR Code, ilustrado em símbolo da Versão 7 \cite{chines}.} 
\end{minipage}
%\vspace{0.1cm}

\section{Metodologia} \label{metodologia}

\subsection{Coleta das imagens QR Code}
Foram gerados 13 QR Codes, de versões e níveis de correção de erro distintos (Veja Tabela \ref{qrtable}), que foram impressos em folhas A4, sendo 4 imagens por página só frente, fato que influencia no tamanho da imagem a ser coletada, por conseguinte, em sua resolução. Em seguida, buscou-se locais com diferentes padrões de luminosidade, a fim de simular variadas situações problema enfrentadas pela tecnologia de reconhecimento.

\vspace{0.2cm}
\begin{minipage}[h]{\columnwidth}
\begin{scriptsize}
    \def\arraystretch{1.35}
    \captionsetup{type=table}
    \begin{center}
    \caption{QR Codes gerados.} \label{qrtable}
    \vspace{-0.2cm}
    \begin{tabular}{ c  c   c} \hline
    \textbf{Nome da Imagem}&  \textbf{Versão}&  \textbf{Nível de Correção de Erro}\\\hline
    QR Code 1  & 3 & L \\
    QR Code 2  & 4 & L \\
    QR Code 3  & 5 & L \\
    QR Code 4  & 6 & M \\
    QR Code 5  & 8 & H  \\
    QR Code 6  & 8 & Q  \\
    QR Code 7  & 10 & H  \\
    QR Code 8  & 12 & H  \\
    QR Code 9  & 13 & H  \\
    QR Code 10 & 17 & H  \\
    QR Code 11 & 29 & L  \\
    QR Code 12 & 38 & H \\
    QR Code 13 & 39 & Q \\\hline
    \end{tabular}
    \end{center}
\end{scriptsize}
\end{minipage}
\vspace{0.5cm}

Escolhidos três ambientes, utilizou-se a camêra de um celular (\emph{Samsung SM-J730G}) de $13\ Mpx$ para realizar a captura dos QR Codes impressos. Assim, obteve-se 39 imagens no formato \emph{jpg} de resolução  $4128 \times 3096$. 
Na Figura \ref{exemplo} há um exemplo de coleta de imagens de um QR Code em específico, QR Code 3 conforme a Tabela \ref{qrtable}, submetido a três padrões de iluminação. 

Essas imagens exemplo serão base para a discussão e demonstração das técnicas de binarização propostas neste artigo. Nas Figuras \ref{exemplo}(a) e (b), é evidente que QR Codes expostos a iluminações mais uniformes tendem a apresentar histogramas com dois picos, um representando os módulos mais escuros e o outro, os mais claros. Constata-se ainda, nas Figuras \ref{exemplo}(c) e (d), que imagens advindas de uma iluminação aproximadamente em degradê dispõem de histogramas com mais picos. Agrega-se que a imagem da Figura \ref{exemplo}(e) resulta em um histograma também com dois picos, mas, em virtude da baixa intensidade nível de cinza médio, os píxeis concentram-se próximos ao início da escala de cinza, como mostra a Figura \ref{exemplo}(f).


\subsection{Procedimento realizado sobre as capturas de imagem}
No fluxograma da Figura \ref{fluxograma}, visualiza-se o procedimento realizado sobre cada imagem capturada, a fim de se obter os dados da Seção \ref{resultados}. Após a captura da imagem, é realizada a transformação para a escala de cinza, para logo depois reduzir ainda mais o espectro de cores da imagem mediante a aplicação do método de binarização (Veja Seção \ref{metodos}). Feito isso, é possível a comparação com o código QR gerado, o que permite gerar as imagens diferença, nas quais os píxeis de divergência entre a imagem gerada e a capturada são marcados na cor vermelha e, em seguida, contabilizados de maneira percentual para cada método desenvolvido e descrito na Seção \ref{metodos}.

\hspace{-0.365cm} 
\begin{minipage}[h]{\columnwidth}
\captionsetup{type=figure}
\subfloat[Iluminação 1]{\includegraphics[width=3.5cm]{imagem03}}\hfill
\subfloat[Histograma de (a)]{\includegraphics[width=4.52cm]{imagem03-hist}}\\
\subfloat[Iluminação 2]{\includegraphics[width=3.5cm]{imagem16}}\hfill
\subfloat[Histograma de (c)]{\includegraphics[width=4.52cm]{imagem16-hist}}\\
\subfloat[Iluminação 3]{\includegraphics[width=3.5cm]{imagem29}}\hfill
\subfloat[Histograma de (e)]{\includegraphics[width=4.52cm]{imagem29-hist}}
\vspace{-0.05cm}
\caption{Exemplos de imagens de um mesmo QR Code nos três diferentes padrões de iluminação e seus respectivos histogramas.}
\label{exemplo}
\end{minipage}

\vspace{0.5cm}
\begin{minipage}[h]{\columnwidth}
\centering
\captionsetup{type=figure}
\includegraphics[scale=0.58]{Fluxograma}
\caption{Fluxograma do procedimento realizado sobre cada imagem.} \label{fluxograma}
\end{minipage}
%\vspace{0.1cm}

\section{Métodos de Binarização de QR Code} \label{metodos}
A \emph{binarização} ou \emph{limiarização} é uma função de transformação de intensidade baseadas em pontos de corte ou limiares. Dada uma imagem em nível de cinza $f(x, y)$, a binarização, no caso da utilização de somente um limiar $T$ (do inglês \emph{Threshold}), resultará em uma imagem $g(x, y)$  de dois níveis ou binária, descrita matematicamente pela Equação (\ref{Threshold}).

\begin{gather}
g( x,\ y) =\begin{cases}
0 & \quad \quad \text{se }\ f( x,\ y) \leqslant T\\
255 & \quad \quad \text{se }\ f( x,\ y)  >T
\end{cases}
\label{Threshold}
\end{gather}
\vspace{0.1cm}

onde 0 e 255 são, respectivamente, as cores preta e branca para imagens em níveis de cinza codificadas em 8 bits.\\

São propostos, neste trabalho, os seguintes métodos: \emph{Binarização por Limiar}; \emph{Binarização Pós Equalização por Histograma}; \emph{Binarização em Sub-Regiões}. A seguir, são descritos os métodos desenvolvidos e utiliza-se imagens de um mesmo QR Code para melhor compreensão, as quais apresentam interessantes condições de iluminação para análise (Veja Figura \ref{exemplo}).


\subsection{Binarização por Limiar} \label{Aconst}
O primeiro método proposto neste artigo consiste na binarização da imagem a partir de dois limiares distintos. Um fixo $T_1=128$, valor correspondente à metade da escala de cinza (\textit{Método do Limiar Fixo - LF}), e um variável, obtido a partir da intensidade média de nível de cinza da imagem analisada, $T_2=M$  (\textit{Método do Limiar Variável - LV}). Portanto, tem-se as funções de transformação descritas, respectivamente, pelas Equações (\ref{eqLF}) e (\ref{eqLV}).


\begin{gather}
\vspace{-1cm}
g_{LF}( x,\ y) =\begin{cases}
0 & \quad \quad \text{se }\ f( x,\ y) \leqslant 128\\
255 & \quad \quad \text{se }\ f( x,\ y)  >128
\end{cases}
\label{eqLF}
\end{gather}

%\vspace{0.02cm}
\begin{gather}
g_{LV}( x,\ y) =\begin{cases}
0 & \quad \quad \text{se }\ f( x,\ y) \leqslant M\\
255 & \quad \quad \text{se }\ f( x,\ y)  >M
\end{cases}
\label{eqLV}
\end{gather}
%\vspace{0.1cm}

Aplicando-se essas tranformações no QR Code 3 (Veja Tabela \ref{qrtable}) obteve-se as imagens das Figuras \ref{figLF} e \ref{figLV}, nas quais verifica-se a eficiência do método sobre padrões de iluminação mais uniformes, visualizando-se os píxeis de erro das imagens diferença ou na Tabela \ref{tabela-erros}.



\vspace{0.25cm}
Observe, na Figura \ref{figLF}(e), que a técnica \emph{LF} sobre a imagem no padrão de Iluminação 3 obteve resultado desvantajoso, uma vez que a baixa iluminação do local levou ao escurecimento total da imagem, o que invialibiliza o reconhecimento do símbolo. Infere-se, assim, que a intensidade de todos os pixeis da imagem são inferiores a 128. Já na Figura \ref{figLV}(e), a técnica \emph{LV} revelou menor erro para esse padrão e, agora, é visível o símbolo QR (Veja Tabela \ref{tabela-erros}). 

Entretanto, não existe relação fixa de eficácia entre os métodos de binarização por limiar, uma vez que nem sempre a aplicação do método \emph{LV} trará resultado mais favorável que o \emph{LF}. Deve-se considerar que os píxeis pretos intrínsecos ao código afetam na determinação da intensidade média $M$ relativa à imagem, por conseguinte, quanto maior o tamanho do símbolo, ou seja maior número de versão, maior será a quantidade de píxeis escuros instrínsecos ao código que influenciará o valor $M$. Para essas situações \emph{LF} pode ser método ótimo ou pelo menos superior a \emph{LV}, o que é enfatizado nos resultados obtidos para as imagens do segundo padrão de QR Codes de versões mais elevadas (Veja Tabela \ref{tabP2}).

\hspace{-0.43cm}
\begin{minipage}[h]{\columnwidth}
\captionsetup{type=figure}
\subfloat[]{\includegraphics[width=3.68cm]{imagem03-LF}} \hfill
\subfloat[]{\includegraphics[width=3.68cm]{imagem03-LF-diff}}\\
\subfloat[]{\includegraphics[width=3.68cm]{imagem16-LF}}\hfill
\subfloat[]{\includegraphics[width=3.68cm]{imagem16-LF-diff}}\\
\subfloat[]{\includegraphics[width=3.68cm]{imagem29-LF}} \hfill 
\subfloat[]{\includegraphics[width=3.68cm]{imagem29-LF-diff}}
\vspace{-0.1cm}
\caption{Imagens resultantes da Binarização com Limiar Fixo para o QR Code 3 nos padrões de (a) Iluminação 1, (c) Iluminação 2 e (e) Iluminação 3, com suas respectivas imagens diferença em (b), (d) e (f).} \label{figLF}
\end{minipage}

\subsection{Binarização Pós Equalização por Histograma (EH)} \label{Bhist}
Outra técnica de binarização proposta por este artigo consiste em duas etapas. Realiza-se, primeiramente, a equalização por histograma da imagem do código QR, por meio da função \emph{histeq()} inclusa no pacote \emph{image} do \emph{Octave}, para, em seguida, aplicar a binarização com limiar correspondente a intensidade média de nível de cinza  $M$ da imagem equalizada. Na Figura \ref{figEH} estão os resultados proporcionados por essa técnica.

De acordo com Gonzalez e Woods \cite{gonzales}, a equalização de histograma tem como objetivo redistribuir os valores de níveis de cinza do píxeis, de forma a obter uma melhora no contraste da imagem. Assim, a equalização da imagem resultará em uma intensidade média $M$ distinta, a qual pode adequar-se melhor ou não aos propósitos da binarização. Na Figura \ref{imagemEH}, observa-se o espalhamento das barras do histograma sobre a escala de cinza após a equalização. Nesse exemplo é utlizada a imagem do QR Code 1 no padrão de Iluminação 2, na qual a técnica \emph{EH} obteve resultado ótimo (Veja Tabela \ref{tabP2}).


\hspace{-0.38cm}
\begin{minipage}[h]{\columnwidth}
\captionsetup{type=figure}
\subfloat[]{\includegraphics[width=3.68cm]{imagem03-LV}} \hfill
\subfloat[]{\includegraphics[width=3.68cm]{imagem03-LV-diff}} \\
\subfloat[]{\includegraphics[width=3.68cm]{imagem16-LV}}\hfill
\subfloat[]{\includegraphics[width=3.68cm]{imagem16-LV-diff}}\\
\subfloat[]{\includegraphics[width=3.68cm]{imagem29-LV}} \hfill 
\subfloat[]{\includegraphics[width=3.68cm]{imagem29-LV-diff}}
\vspace{-0.1cm}
\caption{Imagens resultantes da Binarização com Limiar Variável para o QR Code 3 nos padrões de (a) Iluminação 1, (c) Iluminação 2 e (e) Iluminação 3, com suas respectivas imagens diferença em (b), (d) e (f).} \label{figLV}
\end{minipage}

\vspace{0.3cm}
\hspace{-0.38cm}
\begin{minipage}[h]{\columnwidth}
\captionsetup{type=figure}
\subfloat[]{\includegraphics[width=3.5cm]{imagem14}}\hfill
\subfloat[]{\includegraphics[width=4.52cm]{imagem14-hist}}\\
\subfloat[]{\includegraphics[width=3.5cm]{imagem14-EH-primaria}}\hfill 
\subfloat[]{\includegraphics[width=4.52cm]{imagem14-EH-hist}} 
\vspace{-0.05cm}
\caption{(a) QR Code 1 submetido ao padrão de Iluminação 2, (b) histograma de (a), (c) imagem resultante da equalização de histograma e (d) histograma de (c).}
\label{imagemEH}
\end{minipage}



\hspace{-0.38cm}
\begin{minipage}[h]{\columnwidth}
\captionsetup{type=figure}
\subfloat[]{\includegraphics[width=3.68cm]{imagem03-EH}} \hfill
\subfloat[]{\includegraphics[width=3.68cm]{imagem03-EH-diff}} \\
\subfloat[]{\includegraphics[width=3.68cm]{imagem16-EH}}\hfill
\subfloat[]{\includegraphics[width=3.68cm]{imagem16-EH-diff}}\\
\subfloat[]{\includegraphics[width=3.68cm]{imagem29-EH}} \hfill
\subfloat[]{\includegraphics[width=3.68cm]{imagem29-EH-diff}}
\vspace{-0.1cm}
\caption{Imagens resultantes da Binarização Pós Equalização por Histograma para o QR Code 3 nos padrões de (a) Iluminação 1, (c) Iluminação 2 e (e) Iluminação 3, com suas respectivas imagens diferença em (b), (d) e (f).} \label{figEH}
\end{minipage}


\subsection{Binarização em Sub-Regiões (S)} \label{Cdiv} 
O último método proposto neste artigo consiste em, primeiramente, subdividir a imagem em $n$ regiões retangulares, assim cada uma é binarizada utilizando-se a intensidade média $M$ como limiar, para depois unirem-se em uma imagem só binarizada. Escolheu-se $n=9$, quantidade considerada suficiente para observar claramente o efeito desse método sobre as capturas de imagem QR Code. Na Figura \ref{figS} observa-se os resultados obtidos e, da Tabela \ref{tabela-erros}, afirma-se que obteve baixo nível de erro, com excessão da imagem no segundo padrão de iluminação, cuja irregulidade acentuada provocou que, na separação, as regiões mais claras fossem mais afetadas pelos níveis de intensidades mais escuros intrínsecos do código.


\subsection{Análise erros final para as capturas do QR Code 3}
A aplicação dos diferentes métodos de binarização desenvolvidos e a subsequente comparação com o símbolo QR representado nas imagens de cada padrão permitiu a elaboração da Tabela \ref{tabela-erros}. Nela são contemplados os erros a associados a cada método aplicados nos 3 padrões de iluminação. 


% imagem monocromatica = ou em escala de cinza
\hspace{-0.38cm}
\begin{minipage}[h]{\columnwidth}
\captionsetup{type=figure}
\subfloat[]{\includegraphics[width=3.68cm]{imagem03-S}}\hfill
\subfloat[]{\includegraphics[width=3.68cm]{imagem03-S-diff}} \\
\subfloat[]{\includegraphics[width=3.68cm]{imagem16-S}}\hfill
\subfloat[]{\includegraphics[width=3.68cm]{imagem16-S-diff}}\\
\subfloat[]{\includegraphics[width=3.68cm]{imagem29-S}}\hfill
\subfloat[]{\includegraphics[width=3.68cm]{imagem29-S-diff}}
\vspace{-0.1cm}
\caption{Imagens resultantes da Binarização em Sub-regiões para o QR Code 3 nos padrões de (a) Iluminação 1, (c) Iluminação 2 e (e) Iluminação 3, com suas respectivas imagens diferença em (b), (d) e (f).} \label{figS}
\end{minipage}

\vspace{0.5cm}
Percebe-se a concentração dos erros associados em intervalos bem definidos para os padrões Iluminação 1 e 2, uma vez que as intensidades dos píxeis estão melhor distribuídas ao longo da escala de cinza, como pode ser verificado nos histogramas da Figura \ref{exemplo}. Isso não ocorre para o terceiro padrão, devido à sua natureza escura, além disso verifica-se melhores resultados para técnicas que envolvem mais de uma etapa, ou seja, \emph{Métodos EH} e \emph{S}, uma vez que é necessária primeiramente a otimização do contraste da imagem a fim de se obter um limiar médio $M$ mais adequado. 

\vspace{0.3cm}
\begin{minipage}[h]{\columnwidth}
\begin{scriptsize}
\def\arraystretch{1.35}
    \captionsetup{type=table}
    \begin{center}
    \caption{Erros associados às capturas de imagem do QR Code 3} \label{tabela-erros} \vspace{-0.2cm}
    \begin{tabular}{ c  c  c  c } \hline
    \textbf{Método}&  \textbf{Iluminação 1}& \textbf{Iluminação 2}& \textbf{Iluminação 3}\\\hline
     LF & 15,27\% & 20,33\% & 49,89\%\\
     LV & 13,97\% & 20,73\% & 34,73\%\\
     EH & 14,26\% & 20,33\% & 34,24\%\\
     S   & 13,66\% & 21,00\% & 23,43\%\\\hline
    \end{tabular}
   \end{center}
\end{scriptsize}
\end{minipage}


\section{Análise dos resultados} \label{resultados}
Aplicando-se o processo descrito pelo fluxograma da Figura \ref{fluxograma} em todas as imagens coletadas, obtém-se os dados das Tabelas \ref{tabP1}, \ref{tabP2} e \ref{tabP3}, as quais informam o método ótimo para cada imagem e o respectivo erro associado a ele. Além disso, percebe-se melhor aceitacação de métodos com uma só etapa, como \emph{LF} e \emph{LV}, em ambientes mais claros, como o primeiro e segundo padrão de iluminação, da mesma forma que ambientes mais escuros ou com irregularidade expressiva podem requerer pré-processamento, como ocorre nos \emph{Métodos EH} e \emph{S}, com o intuito de uniformizar a imagem e, assim, poder executar a binarização utilizando-se um limiar médio mais adequado.

\vspace{0.5cm}
\begin{minipage}[h]{\columnwidth}
\begin{scriptsize} 
\def\arraystretch{1.35}
\captionsetup{type=table}
\begin{center}
\caption{Erro, em porcentagem, referente a cada método e método ótimo aplicado sobre as imagens de QR Code submetidas ao padrão de Iluminação 1.} \label{tabP1} \vspace{-0.2cm}
\begin{tabular}{c c c c c c}\hline
\textbf{Nome do QR Code}&\textbf{Método LF(\%)}&\textbf{Método LV(\%)}&\textbf{Método EH(\%)}&\textbf{Método S(\%)}&\textbf{Método Ótimo} \\\hline
Imagem 01 & QR Code 1 & 5,45 & LF \\
Imagem 02 & QR Code 2 & 11,14 & LV \\
Imagem 03 & QR Code 3 & 13,66 & S \\
Imagem 04 & QR Code 4 & 14,20 & LV \\
Imagem 05 & QR Code 5 & 11,77 & LF \\
Imagem 06 & QR Code 6 & 21,00 & LF \\
Imagem 07 & QR Code 7 & 30,97 & LV \\
Imagem 08 & QR Code 8 & 35,79 & EH \\
Imagem 09 & QR Code 9 & 7,64 & S\\
Imagem 10 & QR Code 10 & 25,63 & LV \\
Imagem 11 & QR Code 11 & 47,28 & EH\\
Imagem 12 & QR Code 12 & 45,22 & EH \\
Imagem 13 & QR Code 13 & 28,69 & LV\\\hline
\end{tabular}
\end{center}
\end{scriptsize}
\end{minipage}

\vspace{0.5cm}
\begin{minipage}[h]{\columnwidth}
\begin{scriptsize}
\def\arraystretch{1.35}
\captionsetup{type=table}
\begin{center}
\caption{Erro, em porcentagem, referente a cada método e método ótimo aplicado sobre as imagens de QR Code submetidas ao padrão de Iluminação 2.} \label{tabP2} \vspace{-0.2cm}
\begin{tabular}{c c c c c c}\hline
\textbf{Nome do QR Code}&\textbf{Método LF(\%)}&\textbf{Método LV(\%)}&\textbf{Método EH(\%)}&\textbf{Método S(\%)}&\textbf{Método Ótimo} \\\hline
Imagem 14 & QR Code 1 & 10,78 & EH \\
Imagem 15 & QR Code 2 & 7,43 & LF \\
Imagem 16 & QR Code 3 & 20,33 & LF \\
Imagem 17 & QR Code 4 & 19,91 & EH \\
Imagem 18 & QR Code 5 & 37,88 & EH\\
Imagem 19 & QR Code 6 & 22,23 & LF\\
Imagem 20 & QR Code 7 & 25,61 & LF\\
Imagem 21 & QR Code 8 & 24,43 & S \\
Imagem 22 & QR Code 9 & 37,59 & LF \\
Imagem 23 & QR Code 10 & 21,76 & S \\
Imagem 24 & QR Code 11 & 45,61 & S \\
Imagem 25 & QR Code 12 & 35,39 & S \\
Imagem 26 & QR Code 13 & 47,12 & LF\\\hline
\end{tabular}
\end{center}
\end{scriptsize}
\end{minipage}

\vspace{0.5cm}
\begin{minipage}[h]{\columnwidth}
\begin{scriptsize}
\def\arraystretch{1.35}
\captionsetup{type=table}
\begin{center}
\caption{Erro, em porcentagem, referente a cada método e método ótimo aplicado sobre as imagens de QR Code submetidas ao padrão de Iluminação 3.} \label{tabP3} \vspace{-0.2cm}
\begin{tabular}{c c c c c c}\hline
\textbf{Nome do QR Code}&\textbf{Método LF(\%)}&\textbf{Método LV(\%)}&\textbf{Método EH(\%)}&\textbf{Método S(\%)}&\textbf{Método Ótimo} \\\hline
Imagem 27 & QR Code 1 & 10,08 & S \\
Imagem 28 & QR Code 2 & 16,21 & S \\
Imagem 29 & QR Code 3 & 23,43 & S \\ 
Imagem 30 & QR Code 4 & 11,45 & S \\
Imagem 31 & QR Code 5 & 22,38 & S \\
Imagem 32 & QR Code 6 & 20,69 & S \\
Imagem 33 & QR Code 7 & 31,68 & S \\
Imagem 34 & QR Code 8 & 11,75 & EH \\
Imagem 35 & QR Code 9 & 21,33 & EH \\
Imagem 36 & QR Code 10 & 23,97 & S \\
Imagem 37 & QR Code 11 & 41,15 & S \\
Imagem 38 & QR Code 12 & 38,25 & LV \\
Imagem 39 & QR Code 13 & 28,88 & S \\\hline
\end{tabular}
\end{center}
\end{scriptsize}
\end{minipage}

\vspace{0.5cm}
\section{CONCLUSÕES}
A tecnologia de reconhecimento de QR Code enfrenta problemas significativos provocados principalmente pelo ambiente de filmagem, a exemplo disso cita-se a iluminação irregular no momento da captura da imagem. Por isso, é necessária a aplicação de um método de binarização, nesse sentido, no artigo foram desenvolvidos e propostos três alternativas para esse processamento. Submetendo-se capturas de símbolos QR em três distintos padrões de luminosidade aos métodos desenvolvidos, observou-se a melhor eficácia de métodos que levavam em consideração características mais gerais sobre padrões mais claros, como os \emph{Métodos LF} e \textit{LV}, enquanto que para iluminações de irregularidade extrema ou escuras, métodos preconizando etapas otimização de contraste (\emph{Métodos EH}) ou de binarização localizada (\emph{Métodos S}) foram mais eficazes, pois proporcionaram, à binarização, um limiar mais adequado.

%==================================
% REFERÊNCIAS
%==================================
\begin{thebibliography}{9} % apague as linhas abaixo e insira aqui bibliografia

\bibitem{denso}	
    DENSO WAVE Incorporated,
    “Basic Info”. 
    Disponível em \url{http://www.denso-wave.com/en/adcd/fundamental/index.html}. Acesso em jun. 2019.

\begin{comment}
\bibitem{jin-wei}
J. W. Wei, S. G. Dai, P. A. Mu, 
"QR code correction and positioning method based on morphology and Hough transform", 
\emph{Computer and Information
Technology}, Shanghai, China, 2010.

\bibitem{k-suran}
K. Suran, 
"QR Code Image Correction based on Corner Detection and Convex Hull Algorithm". 
\emph{Journal of Multimedia}, Zhejiang, China, 2013.

\bibitem{w-chen}
 W. B. Chen, G. B. Yang, L. Feng, 
"A low complexity image preprocessing method for QR code recognition", 
\emph{Journal of Hunan University}, Hunan, China, 2012.
\end{comment}

\bibitem{chines}
  M. Li, P. Cao, L. Feng, L. Yu, J. Chen, J. Wang,
   “The research of QR code image correction based on image gray feature”,
   \emph{ 2017 First International Conference on Electronics Instrumentation \& Information Systems (EIIS)}, Harbin, China, June 2017.

\bibitem{j-chen}
C. Jin, J. H. Yuan, L. L. Li, E. Y. Chen, G. Han, T. Tang, 
"One of the image correction algorithms of QR code and its implementation
based on wavelet transform".
\emph{2012 IEEE 3rd International Conference on
Software Engineering and Service Science}, Chongqing, China, 2012.


\bibitem{qr-tutorial}
	"QR Code Tutorial", 2015.
	Disponível em \url{https://www.thonky.com/qr-code-tutorial/}. Acesso em jul. 2019.

\bibitem{octave}
"Octave Forge". 
Disponível em \url{https://octave.sourceforge.io/}. Acesso em jul. 2019.

\bibitem{gonzales}	
    R. C. Gonzales, R. E. Woods,
    “Processamento Digial de Imagens”, 
    3. ed. , São Paulo, Pearson Prentice Hall, 2010.

\end{thebibliography}

%--- FIM ---

\end{multicols}
\end{document}
